% This is "aamas2017_sample.tex", a revised version of aamas2016_sample.tex
% This file should be compiled with "aamas2017.cls"
% This example file demonstrates the use of the 'aamas2017.cls'
% LaTeX2e document class file. It is intended for those submitting
% articles to the AAMAS-2017 conference. This file is based on
% the sig-alternate.tex example file.
% The 'sig-alternate.cls' file of ACM will produce a similar-looking,
% albeit, 'tighter' paper resulting in, invariably, fewer pages
% than the original ACM style.
%
% ----------------------------------------------------------------------------------------------------------------
% This .tex file (and associated .cls ) produces:
%       1) The Permission Statement
%       2) The Conference (location) Info information
%       3) The Copyright Line with AAMAS data
%       4) NO page numbers
%
% as against the acm_proc_article-sp.cls file which
% DOES NOT produce 1) through 3) above.
%
% Using 'aamas2017.cls' you don't have control
% from within the source .tex file, over both the CopyrightYear
% (defaulted to 20XX) and the IFAAMAS Copyright Data
% (defaulted to X-XXXXX-XX-X/XX/XX).
% These information will be overwritten by fixed AAMAS 2017  information
% in the style files - it is NOT as you are used to with ACM style files.
%
% ---------------------------------------------------------------------------------------------------------------
% This .tex source is an example which *does* use
% the .bib file (from which the .bbl file is produced).
% REMEMBER HOWEVER: After having produced the .bbl file,
% and prior to final submission, you *NEED* to 'insert'
% your .bbl file into your source .tex file so as to provide
% ONE 'self-contained' source file.
%


\documentclass{aamas2017}

% if you are using PDF LaTeX and you cannot find a way for producing
% letter, the following explicit settings may help

\pdfpagewidth=8.5truein
\pdfpageheight=11truein

\begin{document}

% In the original styles from ACM, you would have needed to
% add meta-info here. This is not necessary for AAMAS 2017  as
% the complete copyright information is generated by the cls-files.


\title{From Misunderstanding to Cooperation: Understanding and Expressing Intentions Through Non-Verbal Actions}

% AUTHORS


% For initial submission, do not give author names, but the
% tracking number, instead, as the review process is blind.

% You need the command \numberofauthors to handle the 'placement
% and alignment' of the authors beneath the title.
%
% For aesthetic reasons, we recommend 'three authors at a time'
% i.e. three 'name/affiliation blocks' be placed beneath the title.
%
% NOTE: You are NOT restricted in how many 'rows' of
% "name/affiliations" may appear. We just ask that you restrict
% the number of 'columns' to three.
%
% Because of the available 'opening page real-estate'
% we ask you to refrain from putting more than six authors
% (two rows with three columns) beneath the article title.
% More than six makes the first-page appear very cluttered indeed.
%
% Use the \alignauthor commands to handle the names
% and affiliations for an 'aesthetic maximum' of six authors.
% Add names, affiliations, addresses for
% the seventh etc. author(s) as the argument for the
% \additionalauthors command.
% These 'additional authors' will be output/set for you
% without further effort on your part as the last section in
% the body of your article BEFORE References or any Appendices.

%\numberofauthors{8} %  in this sample file, there are a *total*
% of EIGHT authors. SIX appear on the 'first-page' (for formatting
% reasons) and the remaining two appear in the \additionalauthors section.
%

\numberofauthors{1}

\author{
% You can go ahead and credit any number of authors here,
% e.g. one 'row of three' or two rows (consisting of one row of three
% and a second row of one, two or three).
%
% The command \alignauthor (no curly braces needed) should
% precede each author name, affiliation/snail-mail address and
% e-mail address. Additionally, tag each line of
% affiliation/address with \affaddr, and tag the
% e-mail address with \email.
% 1st. author
\alignauthor
\# \# \#
%Ben Trovato\titlenote{Dr.~Trovato insisted his name be first.}\\
%       \affaddr{Institute for Clarity in Documentation}\\
%       \affaddr{1932 Wallamaloo Lane}\\
%       \affaddr{Wallamaloo, New Zealand}\\
%       \email{trovato@corporation.com}
% 2nd. author
%\alignauthor
%G.K.M. Tobin\titlenote{The secretary disavows any knowledge of this author's actions.}\\
%       \affaddr{Institute for Clarity in Documentation}\\
%       \affaddr{P.O. Box 1212}\\
%       \affaddr{Dublin, Ohio 43017-6221}\\
%       \email{webmaster@marysville-ohio.com}
% 3rd. author
%\alignauthor Lars Th{\o}rv{\"a}ld\titlenote{This author is the one who did all the really hard work.}\\
%       \affaddr{The Th{\o}rv{\"a}ld Group}\\
%       \affaddr{1 Th{\o}rv{\"a}ld Circle}\\
%       \affaddr{Hekla, Iceland}\\
%       \email{larst@affiliation.org}
}

%\and  % use '\and' if you need 'another row' of author names

% 4th. author
%\alignauthor Lawrence P. Leipuner\\
%       \affaddr{Brookhaven Laboratories}\\
%       \affaddr{Brookhaven National Lab}\\
%       \affaddr{P.O. Box 5000}\\
%       \email{lleipuner@researchlabs.org}

% 5th. author
%\alignauthor Sean Fogarty\\
%       \affaddr{NASA Ames Research Center}\\
%       \affaddr{Moffett Field}\\
%       \affaddr{California 94035}\\
%       \email{fogartys@amesres.org}

% 6th. author
%\alignauthor Charles Palmer\\
%       \affaddr{Palmer Research Laboratories}\\
%      \affaddr{8600 Datapoint Drive}\\
%       \affaddr{San Antonio, Texas 78229}\\
%       \email{cpalmer@prl.com}

%\and

%% 7th. author
%\alignauthor Lawrence P. Leipuner\\
%       \affaddr{Brookhaven Laboratories}\\
%       \affaddr{Brookhaven National Lab}\\
%       \affaddr{P.O. Box 5000}\\
%       \email{lleipuner@researchlabs.org}

%% 8th. author
%\alignauthor Sean Fogarty\\
%       \affaddr{NASA Ames Research Center}\\
%       \affaddr{Moffett Field}\\
%       \affaddr{California 94035}\\
%       \email{fogartys@amesres.org}

%% 9th. author
%\alignauthor Charles Palmer\\
%       \affaddr{Palmer Research Laboratories}\\
%       \affaddr{8600 Datapoint Drive}\\
%       \affaddr{San Antonio, Texas 78229}\\
%       \email{cpalmer@prl.com}

%}

%% There's nothing stopping you putting the seventh, eighth, etc.
%% author on the opening page (as the 'third row') but we ask,
%% for aesthetic reasons that you place these 'additional authors'
%% in the \additional authors block, viz.
%\additionalauthors{Additional authors: John Smith (The Th{\o}rv{\"a}ld Group,
%email: {\texttt{jsmith@affiliation.org}}) and Julius P.~Kumquat
%(The Kumquat Consortium, email: {\texttt{jpkumquat@consortium.net}}).}
%\date{30 July 1999}
%% Just remember to make sure that the TOTAL number of authors
%% is the number that will appear on the first page PLUS the
%% number that will appear in the \additionalauthors section.

\maketitle

\begin{abstract}
Solving situations of misunderstanding requires two abilities: to build a coherent model of others in order to understand them, and to build a model of "me" perceived by others in order to be understood. Having an image of me seen by others requires two recursive orders of modeling, known in psychology as first and second orders of theory of mind.  It becomes especially difficult to find an understanding when agents don’t have a common language to communicate and have to learn and share each other’s intentions through their behaviors. In this paper, we present a cognitive architecture based on both Reinforcement Learning and Inverse Reinforcement Learning that aims to reach mutual understanding in multi-agent scenarios. We study different conditions of empathy that lead to cooperation in prisoner's dilemma.
\end{abstract}

% Note that the category section should be completed after reference to the ACM Computing Classification Scheme available at
% http://www.acm.org/publications/class-2012
% Hint: a useful place to start could be  Computing methodologies →  Artificial intelligence →  Distributed artificial intelligence

% The block below is an example generated by the ACM web page. Replace it with appropriate block for your work.
\begin{CCSXML}
<ccs2012>
<concept>
<concept_id>10010147.10010178.10010219.10010220</concept_id>
<concept_desc>Computing methodologies~Multi-agent systems</concept_desc>
<concept_significance>500</concept_significance>
</concept>
</ccs2012>
\end{CCSXML}

\ccsdesc[500]{Computing methodologies~Multi-agent systems}
% end auto-generated block

\printccsdesc

%Keywords are your own choice of terms by which you would like the paper to be indexed.

\keywords{AAMAS proceedings, \LaTeX, text tagging}

\section{Introduction}

\section{Mutual modelling}

\section{Expressing intentions}

\section{Models of empathy}

\section{Results and Discussion}

\section{Conclusion}


\bibliographystyle{abbrv}
\bibliography{sigproc}  
% sigproc.bib is the name of the Bibliography in this case
% You must have a proper ".bib" file
%  and remember to run:
% latex bibtex latex latex
% to resolve all references
%
% ACM needs 'a single self-contained file'!
%\balancecolumns % GM June 2007
% That's all folks!
\end{document}
